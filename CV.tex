\documentclass{resume} % Use the custom resume.cls style
\usepackage[lining]{libertine}
\usepackage{textcomp}
\usepackage[colorlinks=false,allcolors = magenta]{hyperref}
\usepackage{etaremune,amsfonts}
\usepackage{xcolor}
\newcommand{\MYhref}[3][cyan]{\href{#2}{\color{#1}{#3}}}%
\usepackage{enumitem}
\usepackage{gensymb}
\usepackage[left=0.4 in,top=0.4in,right=0.4 in,bottom=0.4in]{geometry} % Document margins
\newcommand{\tab}[1]{\hspace{.2667\textwidth}\rlap{#1}} 
\newcommand{\itab}[1]{\hspace{0em}\rlap{#1}}
\renewcommand{\baselinestretch}{1}
\name{Kang Lu} % Your name
%\address{123 Pleasant Lane \\ City, State 12345} % Your secondary addess (optional)
\address{+1(317) 487-9570 \\ Kang.LU@du.edu %\\ %\href{https://github.com/Nirvanahz}{Github} %\\ \href{https://kanglu.me}{Website}
}  % Your phone number and email

\begin{document}

\begin{rSection}{Employment}
{\bf University of Denver}, Visiting assistant professor \hfill {Sept. 2020 - Present}
\end{rSection}

\begin{rSection}{Education}
{\bf Indiana University Purdue University Indianapolis}, Ph.D. in Mathematics \hfill {Aug. 2014 - Aug. 2020}

{\bf Zhejiang University}, M.S. in Mathematics \hfill {Sept. 2012 - Jul. 2014}

{\bf Fudan University}, B.S. in Mathematics \hfill {Sept. 2006 - Jul. 2010}
\end{rSection}

\begin{rSection}{Research Interests}
Representation theory, Quantum algebras, and Integrable systems.  
\end{rSection}


%----------------------------------------------------------------------------------------
%	WORK EXPERIENCE SECTION
%----------------------------------------------------------------------------------------



% \begin{rSection}{Experience}
% \begin{rSubsection}{\bf Engineer | Arm  }{Bengaluru, India}
% {\bf Architecture and Technology Engineering Group  }{Jul. 2016 - Sep. 2019}
%   \item {Lead Coverage Model Developer for arm-v8M profile and its extensions.}\\
%   \item {Developed tools to automatically configure A-class Coverage Model to different implementations.}\\
%   \item {Improved coverage development flow in terms of early detection \& root cause analysis of bugs using CI
% best practices.}\\
% \end{rSubsection}

% \begin{rSubsection}{\bf Intern | Arm  }{Bengaluru, India}
% {\bf Architecture and Technology Engineering Group  }{Jun. 2015 - Jul. 2015}
%   \item {Studied coverage model implementation \& regression framework; explored options like Direct
% Programming Interface (DPI), stream mode coverage (less file I/O) to improve regression time.}\\
%   \item {Attained 15\% reduction in memory and 10\% reduction in time for regressions.}\\
% \end{rSubsection}
% \end{rSection}

% %----------------------------------------------------------------------------------------
% \begin{rSection}{Undergraduate Project}
% \begin{rSubsection}{\bf Direction of Arrival Estimation for Lens-Based Array Antennas  }{Bengaluru, India}
% {\bf (Funded by Indian Naval Research Board)  }{Aug. 2015 - May. 2016}
%   \item {Implemented ray-tracing algorithms on MATLAB to simulate refractive effects of various dielectric
% lenses; modified stochastic parameter estimation algorithms (MUSIC, MVDR etc) to account for these effects.}\\
%   \item {Implemented adaptive beamforming algorithms in C++ to direct the beam; achieved a 6dB improvement in
% gain for a 4x4 array antenna with 1\degree resolution and 10dB SNR.}\\
% \end{rSubsection}

% \end{rSection}

\begin{rSection}{Preprints}
\begin{etaremune}[leftmargin=0cm]

\item Kang Lu, {\it Schur-Weyl duality for quantum toroidal superalgebras}, \href{https://arxiv.org/abs/2109.09005}{arXiv:2109.09005}

\item Kang Lu, {\it Gelfand-Tsetlin bases for representations of super Yangian and quantum affine superalgebra}, \href{https://arxiv.org/abs/2103.08758}{arXiv:2103.08758}.

\item Kang Lu, Gang Han, and Jun Yu.
{\it Fine gradings of complex simple Lie algebras and Finite Root Systems}, preprint, \href{https://arxiv.org/abs/1410.7945}{arXiv:1410.7945}. 
		
\end{etaremune}
\end{rSection}

\begin{rSection}{Publications}
\begin{etaremune}[leftmargin=0cm]
\item Kang Lu, E. Mukhin, {\it Jacobi-Trudi identity and Drinfeld functor for super Yangian}, \href{https://dx.doi.org/10.1093/imrn/rnab023}{Int. Math. Res. Not.}

\item Kang Lu, E Mukhin, {\it Bethe ansatz equations for orthosymplectic Lie superalgebras and self-dual superspaces}, \href{https://doi.org/10.1007/s00023-021-01091-8}{Annales Henri Poincar\'{e}}.

\item Kang Lu, E. Mukhin, {\it On the supersymmetric XXX spin chains associated to $\mathfrak{gl}_{1|1}$}, \href{https://dx.doi.org/10.1007/s00220-021-04155-2}{Commun. Math. Phys. \textbf{386} (2021), 711-747}.

\item Kang Lu, {\it Perfect integrability and Gaudin models}, \href{https://doi.org/10.3842/SIGMA.2020.132}{SIGMA {\bf 16} (2020), 132, 10 pages}.

\item Chenliang Huang, Kang Lu, and E. Mukhin.
{\it Solutions of $\mathfrak{gl}_{m|n}$ XXX Bethe ansatz equation and rational difference operators}, \href{https://doi.org/10.1088/1751-8121/ab1960}{J. Phys. A: Math. Theor. \textbf{52} (2019), no. 37, 375204, 31 pages}.	
			
\item Kang Lu, E. Mukhin. 
{\it On the Gaudin model of type G$_2$}, \href{https://doi.org/10.1142/S0219199718500128}{Commun. Contemp. Math. \textbf{21} (2019), no. 3, 1850012, 31 pages}.

\item Gang Han, Yucheng Liu, and Kang Lu. {\it Multiplicity free gradings on semisimple Lie and Jordan algebras and skew root systems}, \href{https://doi.org/10.1142/S1005386719000129}{Algebra Colloq. {\bf 26} (2019), no. 1, 123--138}.
			
\item Kang Lu, {\it Lower bounds for numbers of real self-dual spaces in problems of Schubert calculus}, \href{https://doi.org/10.3842/SIGMA.2018.046}{SIGMA {\bf 14} (2018), 046, 15 pages}.	
			
\item Kang Lu, E. Mukhin, A. Varchenko. 
{\it Self-dual Grassmannian, Wronski map, and representations of $\mathfrak{gl}_N$, $\mathfrak{sp}_{2r}$, $\mathfrak{so}_{2r+1}$}, \href{http://dx.doi.org/10.4310/PAMQ.2017.v13.n2.a4}{Pure Appl. Math. Q. {\bf 13} (2017), no.2, 291--335}, special issue in honor of Yuri Manin's 80-th birthday.
					
\item Kang Lu, E. Mukhin, A. Varchenko. 
{\it On the Gaudin model associated to Lie algebras of classical types}, \href{https://doi.org/10.1063/1.4964389}{J. Math. Phys. {\bf 57} (2016), no. 10, 101703, 23 pages}.
\end{etaremune}
\end{rSection}

\begin{rSection}{Conference Presentations}
\begin{rSubsection}{}{}
{}{}
  \item {2021 AMS Spring Southeastern Sectional Meeting, Special Session on Superalgebras, Quantum Groups, and Related Topics.\\
      \textbf{Talk}: Skew representations of super Yangian}\\
  \item {Joint Mathematics Meetings 2020, Colorado Convention Center, Denver, CO January 15-18, 2020.\\
        \textbf{Talk}: On the supersymmetric XXX spin chains}\\
  \item {\href{http://rtis2019.math.iupui.edu/}{Representation Theory and Integrable Systems}, ETHZ, Zurich, Switzerland, August 12-16, 2019. \\
		{\bf Talk}: On the supersymmetric XXX spin chain associated to $\mathfrak{gl}_{1|1}$}\\
  \item {\href{http://www.ams.org/meetings/sectional/2265_program.html}{2019 AMS Spring Eastern Sectional Meeting}, University of Connecticut, Hartford, CT, April 13-14, 2019. \\
        {\bf Talk}: On the supersymmetric XXX spin chain associated to $\mathfrak{gl}_{1|1}$}\\
  \item {\href{http://www.ams.org/meetings/sectional/2261_program.html}{2019 AMS Spring Southeastern Sectional Meeting}, Auburn University, Auburn, AL, March 15-17, 2019. \\
        {\bf Talk}: Self-dual Grassmannian and Representations of $\mathfrak{gl}_N$, $\mathfrak{sp}_{2r}$, and $\mathfrak{so}_{2r+1}$}\\
  \item {\href{http://reims.math.cnrs.fr/pevzner/aak81.html}{Representation Theory at the Crossroads of Modern Mathematics}, Universit\'{e} de Reims Champagne Ardenne, Reims, France, May 29-June 2, 2017. \\
		{\bf Poster}: \href{https://kanglu.me/writings/poster-Reims.pdf}{Self-dual Grassmannian and Representations of $\mathfrak{gl}_N$, $\mathfrak{sp}_{2r}$, and $\mathfrak{so}_{2r+1}$}}\\
  \item {\href{http://www.ams.org/meetings/sectional/2233_program.html}{2017 AMS Spring Central Sectional Meeting}, Indiana University, Bloomington, April 1-2, 2017. \\
		{\bf Talk}: \href{https://kanglu.me/writings/ams-sec-2017.pdf}{Bethe ansatz method in Gaudin Model}}\\
\end{rSubsection}
\end{rSection}

\begin{rSection}{Seminar Talks}
\begin{rSubsection}{}{}
{}{}
  \item {\href{https://sites.google.com/view/rockymountainreptheory/home}{Rocky Mountain Representation Theory Seminar}, Zoom, March 11, 2021.\\
        {\bf Talk}: Skew representations of super Yangian}\\
  \item {\href{https://research.math.osu.edu/reps/}{Representations and Lie Theory seminar}, @ Ohio State University, Zoom, February 17, 2021.\\
        {\bf Talk}: Skew representations of super Yangian}\\
  \item {Algebra and Logic Seminar, University of Denver, Denver, CO, October 19, 2020. \\
		{\bf Talk}:  Gaudin model, Feigin-Frenkel center, and Grassmannian}\\
  \item {\href{https://math.virginia.edu/seminars/algebra/2019-20/}{Algebra Seminar}, University of Virginia, Charlottesville, VA, November 15, 2019. \\
		{\bf Talk}:  Jacobi-Trudi identity, Berezinian, and transfer matrices}\\
  \item {\href{https://math.virginia.edu/seminars/algebra/2019-20/}{Physically inspired mathematics seminar}, University of North Carolina, Chapel Hill, NC, October 4, 2019. \\
		{\bf Talk}: Supersymmetric quantum spin chains}\\
\end{rSubsection}
\end{rSection}

\begin{rSection}{Teaching}
\begin{rSubsection}{University of Denver}{}
{}{}
  \item {MATH 1951: Calculus I, 2021 Autumn Quarter}\\
  \item {MATH 1952: Calculus II, 2021 Spring Quarter}\\
  \item {MATH 1150: Introduction to Cryptography, 2021 Winter Quarter}\\
  \item {MATH 2070: Introduction to Differential Equations, 2021 Winter Quarter}\\
  \item {MATH 1951: Calculus I, 2020 Autumn Quarter}\\
\end{rSubsection}

\begin{rSubsection}{Indiana University Purdue University Indianapolis}{}
{}{}
  \item {MATH 16500: Calculus and Analytic Geometry I, 2020 Summer I}\\
  \item {MATH 22100: Calculus for Technology I, 2020 Spring}\\
  \item {MATH 15400: Trigonometry, 2019 Fall}\\
  \item {MATH 26600: Ordinary Differential Equations, 2019 Summer II}\\
  \item {MATH 22100: Calculus for Technology I, 2019 Spring}\\
  \item {MATH 11100: Intermediate algebra, 2018 Fall}\\
  \item {MATH 15400: Trigonometry, 2018 Summer II}\\
  \item {MATH 11000: Fundamentals of Algebra, 2018 Spring}\\
  \item {MATH 16500: Calculus and Analytic Geometry I (Recitation), 2017 Fall}\\
\end{rSubsection}
\end{rSection}

\begin{rSection}{Service}
\begin{rSubsection}{}{}
{}{}
\item Referee for: Communications in Mathematical Physics, SIGMA, Transformation Groups \\ 	
\end{rSubsection}
\end{rSection}

\begin{rSection}{References}
\begin{rSubsection}{}{}
{}{}
  \item {Andrew Linshaw, Department of Mathematics, Indiana University of Denver, andrew.linshaw@du.edu}\\
  \item {Evgeny Mukhin, Department of Mathematical Science, Indiana University Purdue University Indianapolis, emukhin@iupui.edu}\\
  \item {Vitaly Tarasov, Department of Mathematical Science, Indiana University Purdue University Indianapolis, vtarasov@iupui.edu}\\
  \item {Alexander Varchenko, Department of Mathematics, University of North Carolina at Chapel Hill, anv@email.unc.edu}\\

\end{rSubsection}
\end{rSection}

\end{document}

