%basic cover letter template
\documentclass[12pt]{amsart} %{letter}
\usepackage{amssymb,amsmath,times}
%\usepackage{helvet}
%\renewcommand{\familydefault}{\sfdefault}
\usepackage{graphicx}
\usepackage[colorlinks=false,citecolor=cyan,linkcolor=blue,pagebackref]{hyperref}
\usepackage{indentfirst}
\usepackage{geometry}
\geometry{verbose,margin=2cm}
\linespread{1.05}
\setlength{\parindent}{1cm}

\begin{document}
\includegraphics{letterhead_official.pdf}


\begin{flushright}
{April 29, 2022}
\end{flushright}
\title[]
{\small Reference Letter for Kang LU}


\maketitle


%\noindent To the Hiring Committee:

\vspace{.3cm}

%\begin{letter}{
%Re: Independent letter for Dr.\ Kang LU\\ 4/23/22
%}

%\date{
%\includegraphics[width=2.5cm]{UVA-Logo.png}
%}

%\opening{
\noindent To Whom it may concern, 
\vspace{2mm}

Based on Dr.\ Kang LU's outstanding publication record, I recognize Dr.\ LU as an extraordinary mathematician in the field. In November 2019, I invited him to give a talk in the algebra similar at University of Virginia (UVA) to discuss his impressive work on Schur-Weyl duality of super Yangian. Prior to this invitation, I never worked with Dr.\ LU before. 

Before proceeding further, let me briefly discuss my credentials. My name is Weiqiang Wang, a full professor at University of Virginia since 2006 and Gordon Whyburn Professor since 2020. I obtained my Ph.D. degree in Mathematics from MIT in 1995. My textbook book ``Dualities and Representations of Lie Superalgebras" (joint with S.-J. Cheng) was published by American Mathematical Society in 2012. I have published over a hundred of peer-reviewed research papers in Mathematics in the past decades. In 2018, I was selected as a fellow of American Mathematical Society, for my contributions to Lie theory and representation theory and service to the mathematical community. I was awarded (joint with my student Huanchen Bao) the 2020 Chevalley Prize of American Mathematical Society, for fundamental contributions to the theory of quantum symmetric pairs. I am an invited speaker of the incoming 2022 International Congress of Mathematicians, the most significant meeting on pure and applied mathematics (held once every 4 years) as well as one of the world's oldest scientific congresses. I have common research interests with Dr.\ LU mainly on representation theory of Lie superalgebras and quantum groups, and partially on mathematical physics.  I believe I am in a good position to judge the work of Dr.\ Kang LU independently and objectively. 

Groups and algebras are mathematical languages used to describe symmetries in the physical world as well as in a more abstract sense. Representation theory is an area which uses matrices in linear algebra to represent groups or algebras so we can understand the structures of the given groups or algebras better. Quantum groups were a class of new algebras introduced by Drinfeld and Jimbo in 1986, for which Drinfeld was awarded a Fields medal (the mathematical equivalent of a Nobel prize). They have played fundamental roles in mathematics and physics. 

First, let me discuss Dr.\ LU's remarkable contribution on Schur-Weyl duality of quantum groups. As one of the most beautiful classical results in representation theory, Schur-Weyl duality connects the representations of the general linear Lie algebras and representations of the symmetric groups. It plays a prominent role in modern mathematics and has numerous applications in algebra, geometry, combinatorics, and mathematical physics. Since the introduction of quantum groups in the 1980s, mathematicians have done work to generalize Schur-Weyl duality in the quantum setting. The picture for supersymmetric case was largely missing until recently greatly enhanced by Dr.\ LU's important work on Schur-Weyl duality for super Yangians and quantum toroidal superalgebras. 


Dr.\ LU novelly extended Drinfeld functor to establish the Schur-Weyl duality for super Yangians. With Dr.\ LU's results, a systematic and accessible approach toward the basic problems of representation theory of super Yangians was made available in a great generality. Dr.\ LU's work also exhibited further remarkable properties of Drinfeld functor. More specifically, Dr.\ LU proved that the Drinfeld functor is exact and sends irreducible representations of super Yangian to irreducibles of degenerate affine Hecke algebra. Moreover, Dr.\ LU enabled the Schur-Weyl duality to convert results about standard Yangian to the new results about super Yangian. Since the representation theory of super Yangian is dramatically different from that of standard Yangian, Dr.\ LU's approach is highly valuable and innovative. 


Let me mention a few of important applications of Dr.\ LU's work on Schur-Weyl duality and Drinfeld functor. For one, by exploiting Schur-Weyl duality, Dr.\ LU provided sufficient conditions for a tensor product of representations of super Yangian to be irreducible, an important topic in representation theory. 
%In particular, he established the binary property of representations of super Yangian that is rather radical to study further involved representations. 
By further exploring properties of Drinfeld functor, Dr.\ LU obtained solutions of the T-systems in terms of super Yangian. The T-systems have turned out to be ubiquitous structures with a wealth of applications in mathematics and physics. For instance, Dr.\ LU's solutions of T-systems allow to study mutations of cluster superalgebras since the equations of a T-system are exactly of the same form as the exchange relations in cluster superalgebras. As a further application, Dr.\ LU's solutions of T-systems can be used to investigate the connection problem of 1D Schr\"{o}dinger equation, which is a typical example of the ODE (ordinary differential equations)/IM (integrable models) correspondence.  

Another application of Dr.\ LU's creative work on Schur-Weyl duality is in physics. Dr.\ LU proved the celebrated Jacobi-Trudi identity for super Yangian using Schur-Weyl duality and obtained a proof of the corresponding identity for transfer matrices that was previous widely used by physicists but without proofs. The proof completed by Dr.\ LU has been highlighted as one of the most important results in the field of quantum integrability in a recent review article by Ryan from King's College London in UK. Dr.\ LU gave a seminar talk at University of Virginia on his prominent proof. Some colleagues in my mathematics department  spoke highly of Dr.\ LU's work afterwards.


Besides the Schur-Weyl duality for super Yangians, Dr.\ LU also initiated the study of Schur-Weyl duality between quantum toroidal superalgebra, a much more sophisticated quantum group than super Yangian, and double affine Hecke algebra. He ingeniously established the equivalence between certain important categories over these two central algebras in the area of representation theory. Similar to the applications of Schur-Weyl duality for super Yangian, his results on Schur-Weyl duality for quantum toroidal superalgebra allows mathematicians and physicists to obtain statements about quantum toroidal superalgebra straightforwardly from statements about the normal quantum toroidal algebra. These results provide the generalizations of aforementioned applications to quantum toroidal superalgebra case.


Next, let me highlight Dr.\ LU's significant work on skew representations. Skew representations are a family of important representations that are associated with fundamental combinatorial objects called skew Young diagrams. It plays a central role in both mathematics and physics. For instance, the transfer matrices from the study of quantum integrable models are described in terms of skew representations and universal R-matrix. Being a generalization of the standard Yangian, the super Yangian is far less studied, and in particular, the part of skew representations for super Yangian was totally unexplored.  Dr.\ LU constructed the skew representations from the scratch and investigated their important properties. He showed how skew representations are related to Schur-Weyl duality via the aforementioned Jacobi-Trudi identity. 


Another important property of skew representations is that they can also be obtained from the R-matrix constructions, which are closely related to quantum integrable models. Indeed, Dr.\ LU managed to provide 3 different approaches to construct skew representations, namely by restrictions, by Schur-Weyl duality, and by R-matrix construction. Dr.\ LU proved that all three approaches eventually give rise to the same structure as a representation of super Yangian. Dr.\ LU's results are highly nontrivial, and one should not take them for granted, because one would not expect that objects obtained via different ways are actually the same thing. Dr.\ LU's R matrix construction of skew representations has been used by Professor A. Molev from University of Sydney, a world's leading expert in representation theory and Yangian, to show that the fundamental representations of super Yangians of type C are finite-dimensional. %Molev's results were further used to classify finite-dimensional irreducible representations of super Yangians of type C.

Dr.\ Kang LU investigated a distinguished basis of skew representations, called Gelfand-Tsetlin basis, and came up with the explicit actions of the generators of super Yangian. These actions can be expressed in terms of combinatorics and are readily accessible by other mathematicians. Dr.\ LU's result was used by Donnelly from Murray State University at Kentucky, to obtain Gelfand–Tsetlin type weight bases for skew representations and investigate the combinatorial and representation-theoretic applications. In particular, the relations of skew Schur functions and their associated skew-tabular lattices from Donnelly's work were established using Dr.\ LU's result. Dr.\ LU's results on Gelfand-Tsetlin basis were further used to study the sufficient and necessary criterion for a tensor product of two skew representations being irreducible. The progress is a major step for fundamental research towards the irreducibility of tensor products of two irreducible Yangian modules.


In summary, through original scientific research, Dr.\ LU has significantly advanced the theory in the area of representation theory and quantum groups, and furthered the pursuit of progress in his discipline with applications to mathematical physics. I am convinced that Dr.\ LU's breakthrough has exhibited theoretical application in advancing the most fundamental research in mathematics and physics and will help improve the basic understanding of nature and the world. We in the field of representation theory and quantum groups have found his contribution invaluable and indispensible. As a fellow scholar, I urge the United States to make all necessary concessions to ensure that Dr.\ LU is able to continue his research. 


\vspace{.5cm}

 \includegraphics{WW-Signature.png}

%\qquad\qquad\qquad\qquad\qquad\qquad\qquad\qquad\qquad\qquad\qquad
Weiqiang Wang

%\qquad\qquad\qquad\qquad\qquad\qquad\qquad\qquad\qquad\qquad\qquad
Gordon Whyburn Professor of Mathematics


% \closing{Sincerely,	%\includegraphics[width=2cm]{signature.pdf}
% }
%I made a pdf file of my signature using the scanner, so that all the cover letters I sent
%electronically were "signed."  

%\vspace{-2.5cm}
% \noindent Weiqiang Wang\\
% Department of Mathematic\\
% University of Virginia

%\end{letter}

\end{document}